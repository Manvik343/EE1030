%iffalse
\let\negmedspace\undefined
\let\negthickspace\undefined
\documentclass[journal,12pt,twocolumn]{IEEEtran}
\usepackage{cite}
\usepackage{amsmath,amssymb,amsfonts,amsthm}
\usepackage{algorithmic}
\usepackage{graphicx}
\usepackage{textcomp}
\usepackage{xcolor}
\usepackage{txfonts}
\usepackage{listings}
\usepackage{enumitem}
\usepackage{mathtools}
\usepackage{gensymb}
\usepackage{comment}
\usepackage[breaklinks=true]{hyperref}
\usepackage{tkz-euclide} 
\usepackage{listings}
\usepackage{gvv}                                        
%\def\inputGnumericTable{}                                 
\usepackage[latin1]{inputenc}                                
\usepackage{color}                                            
\usepackage{array}                                            
\usepackage{longtable}                                       
\usepackage{calc}                                             
\usepackage{multirow}                                         
\usepackage{hhline}                                           
\usepackage{ifthen}                                           
\usepackage{lscape}
\usepackage{tabularx}
\usepackage{array}
\usepackage{float}

\usepackage{enumitem}
\usepackage{xcolor}
\usepackage{multicol}


\newtheorem{theorem}{Theorem}[section]
\newtheorem{problem}{Problem}
\newtheorem{proposition}{Proposition}[section]
\newtheorem{lemma}{Lemma}[section]
\newtheorem{corollary}[theorem]{Corollary}
\newtheorem{example}{Example}[section]
\newtheorem{definition}[problem]{Definition}
\newcommand{\BEQA}{\begin{eqnarray}}
\newcommand{\EEQA}{\end{eqnarray}}
\newcommand{\define}{\stackrel{\triangle}{=}}
\theoremstyle{remark}
\newtheorem{rem}{Remark}

\title{Section-B JEE Main-Quadratic Equations}
\author{AI24BTECH11021 - Manvik Muthyapu}
\begin{document}
\bibliographystyle{IEEEtran}
\maketitle
\newpage
\bigskip

\renewcommand{\thefigure}{\theenumi}
\renewcommand{\thetable}{\theenumi}

\begin{enumerate}

%8
\item The number of real solutions of the equation \\
	$ x^2 - 3\abs{x} + 2 = 0$ is

\begin{enumerate}

	\item  $3$
	\item  $2$
	\item  $4$
	\item  $1$
    
\end{enumerate}

%9
\item The real number $x$ when added to its inverse gives the minimum value of the sum at $x$ equal to
\hfill[2003]

\begin{enumerate}

	\item  $-2$
	\item  $2$
	\item  $1$
	\item  $-1$

\end{enumerate}

%10
\item Let two numbers have arithmetic mean $9$ and geometric mean $4$. Then these numbers are the roots of the quadratic equation
\hfill[2004]

\begin{enumerate}

	\item  $x^2 - 18x - 16 = 0$
	\item  $x^2 - 18x + 16 = 0$
	\item  $x^2 + 18x - 16 = 0$
	\item  $x^2 + 18x + 16 = 0$

\end{enumerate}

%11
\item If $(1 - p)$ is a root of quadratic equation\\
$x^2 + px + (1 - p) = 0$ then its root are
\hfill[2004]

\begin{enumerate}

	\item  $-1,2$
	\item  $-1,1$
	\item  $0,-1$
	\item  $0,1$

\end{enumerate}

%12
\item If one root of the equation $x^2 + px + 12 = 0$ is $4$, while the equation  $x^2 + px + q = 0$ has equal roots , then the value of `$q$' is
\hfill[2004]

\begin{enumerate}

	\item  $4$
	\item  $12$
	\item  $3$
	\item  $\frac{49}{4}$

\end{enumerate}

%13
\item In a triangle $PQR$, $\angle R = \frac{\pi}{2}$. If $ \tan\brak{\frac{P}{2}}$ and \\
	$- \tan\brak{\frac{Q}{2}}$ are the roots of $ax^2 + bx + c = 0$, $a \neq 0$ then
\hfill[2005]

\begin{enumerate}


	\item  $a = b + c$
	\item  $c = a + b$
	\item  $b = c$
	\item  $b = a + c$

\end{enumerate}

%14
\item If both the roots of the quadratic equation  $x^2 - 2kx + k^2 + k - 5 = 0$ are less than $5$, then $k$ lies in the interval
\hfill[2005]

\begin{enumerate}

	\item  $(5,6]$
	\item  $(6,\infty)$
	\item  $(- \infty,4)$
	\item  $\sbrak{4,5}$

\end{enumerate}

%15
\item If the roots of the quadratic equation $x^2 + px + q = 0$ are $\tan 30\degree$ and $\tan 15\degree$, respectively, then the value of $2 + q - p$ is
\hfill[2006]

\begin{enumerate}


	\item  $2$
	\item  $3$
	\item  $0$
	\item  $1$
		
\end{enumerate}

%16
\item All the values of $m$ for which both roots of the equation $x^2 - 2mx + m^2 - 1 = 0$ are greater than $-2$ but less than $4$, lie in the interval
\hfill[2006]

\begin{enumerate}

	\item  $-2 < m < 0$
	\item  $m > 3$
	\item  $-1 < m < 3$
	\item  $1 < m < 4$

\end{enumerate}

%17
\item If $x$ is real, the maximum value of\\
$\frac{3x^2 + 9x + 17}{3x^2 + 9x + 7}$ is
\hfill[2006]

\begin{enumerate}

	\item  $\frac{1}{4}$
	\item  $41$
	\item  $1$
	\item  $\frac{17}{7}$

\end{enumerate}

%18
\item If the difference between the roots of the equation $x^2 + ax + 1 = 0$ is less than $\sqrt{5}$, then the set of possible values of $a$ is
\hfill[2007]

\begin{enumerate}

	\item  $(3,\infty)$
	\item  $(-\infty, -3)$
	\item  $(-3, 3)$
	\item  $(-3, \infty)$

\end{enumerate}

%19
\item \textbf{Statement-1 :} For every natural number $n\geq 2$,\\
$$\frac{1}{\sqrt{1}} + \frac{1}{\sqrt{2}} + ......... + \frac{1}{\sqrt{n}} > \sqrt{n}$$\\
\textbf{Statement-2 :} For every natural number $n\geq 2$,\\
$$\sqrt{n(n + 1)} < n + 1$$
\hfill[2008]\\

\begin{enumerate}

	\item  Statement-$1$ is false, Statement-$2$ is true
	\item  Statement-$1$ is true, Statement-$2$ is true, Statement-$2$ is a correct explanation for Statement-$1$
	\item  Statement-$1$ is true, Statement-$2$ is true, Statement-$2$ is not a correct explanation for Statement-$1$
	\item  Statement-$1$ is true, Statement-$2$ is false

\end{enumerate}

%20
\item The quadratic equations $x^2 - 6x + a = 0$ and $x^2 - cx + 6 = 0$ have one root in common. The roots of the first and second equations are integers in the ratio $4 : 3$. Then the common root is
\hfill[2009]

\begin{enumerate}

	\item  $1$
	\item  $4$
	\item  $3$
	\item  $2$

\end{enumerate}

%21
\item If the roots of the equation $bx^2 + cx + a = 0$ be imaginary, then for all the real values of $x$, the expression\\
$3b^2x^2 + 6bcx + 2c^2$ is :
\hfill[2009]

\begin{enumerate}

	\item  less than $4ab$
	\item  greater than $-4ab$
	\item  less than $-4ab$
	\item  greater than $4ab$

\end{enumerate}

%22
\item If $\abs{z - \frac{4}{z}} = 2$ , then the maximum value of $\abs{Z}$ is equal to :
\hfill[2009]

\begin{enumerate}

	\item  $\sqrt{5} + 1$
	\item  $2$
	\item  $2 + \sqrt{2}$
	\item  $\sqrt{3} + 1$

\end{enumerate}

%23
\item If $\alpha$ and $\beta$ are the roots of the equation $x^2 - x + 1 = 0$, then $\alpha^{2009} + \beta^{2009} =$
\hfill[2010]

\begin{enumerate}

	\item  $-1$
	\item  $1$
	\item  $2$
	\item  $-2$

\end{enumarte}

\end{enumerate}
\end{document}
